\documentclass{article}
\usepackage{graphicx}
\graphicspath{ {./Images/} }
\usepackage[utf8]{inputenc}

\title{STRATEGIES FOR IMPROVING SMALL SCALE ENTERPRISES}
\author{Vibhanshu Singh}
\date{February 2022}

\begin{document}

\maketitle

\section{Abstract}
\section{Introduction}
\section{Strategies for the Growth of Small-Scale Enterprise}
\section{Expansion}
\section{Diversification}
\section{Joint venture}
\section{Mergers and Acquisitions}

\newpage

\section*{Abstract}
\newcommand\shortlorem{The study is aimed at the reappraising on the strategies for improving small scale enterprises. The study would help to identify some of the strategies for improving small scale enterprise. It will also help the future investors to be aware of the strategies to be encountered. However, the research question are based on the strategies for improving small scale enterprise, the roles of the government to play in improving small scale enterprises such as providing of basic infrastructural facilities, providing of adequate providing of adequate capital or finance. During the research necessary data was collected through the use of questionnaire and was analysed making use of tabulation method and simple percentage. During the course of the project the survey research method was the research method used. Moreover, the pilot survey was used for the sample size while simple random are used for the sampling technique i.e. thirty five out of forty of the staff were randomly selected. Also effort were made by the Researchers to collect the needed data and accurate information through the use of both primary and secondary sources of data. The data collected was analysed, interpreted and presented using tabulation method and simple percentage. Furthermore the researchers found out that lack of fund, inadequate planning, incompetence of management, poor competitive position are the problems facing small scale enterprises. Also they found out that the government, individual and other organization over the environment have a great role to play in the management of small scale establishment in our country. Finally the researchers recommended that appropriate credit guarantee schemes should be established and the support from credit bureau developed. They also recommended that Development of Infrastructure by the government should be provided to support small scale enterprises.}

\shortlorem



\section*{Introduction}
 


The strategies for improving small scale enterprise in Nigeria is a programme embarked
upon by developing countries as one of the asset of economic increase of their nations.
Before Nigeria had their independence in 1960, the people of Nigeria were engaged in
commerce and in various kinds of occupations like fishing, hunting, buying and selling to
earn their living. These occupations were determined by factors such as geographical
Location and Nature of land, etc.However, they engage on those activities but there were a lot of problems facing them
such as, under-capital, inadequate infrastructure, lack of experience, etc. And these led to
the introduction of a programme known as (SAP) Structural Adjustment programme in
July 1986 by the government of General Ibrahim Badanasi Babangida. This programme
was to help in generating the Nigeria Economy. It went a long way in encouraging private
individuals to participate in the ownership of their own companies.The structural adjustment programme (SAP) Introduced made employment rate to decline
as a result of placing an embargo on employment, retrenchment of workers followed and
many labours both skilled and unskilled could not get employment decided to get a self
employed business such as craft work, drawings and paintings etc.The small scale enterprises is a private owned and operated business characterized by a
small number of employees and low turnover.They are normally private owned co-operations and sole-proprietorship. Small scale
enterprise varies country by country and industry by industry. Taking for instance ina under the work act of 2006; The range is fever than 15 employees. In Europe,
their range is not more than 50employees and fewer than 500 employees to qualify for
U.S small business Administration programs. (Isemin 2009:19) The importance of small scale business enterprise includes the following: small scale Download
enterprise promote competition and hinder monopoly. Small scale enterprise serves as a
nursery for entrepreneurial organization all over the world. They ensure the supply of Pownload
goods and services within the reach of a great majority of the populations. Finally they
employ labour intensive technology and also provides further, the necessary
apprenticeship training in skills and entrepreneurship which enables the beneficiaries get Make Money
self employment. The small scale business which enjoys the patronage of the overall investors were established for the purpose of retailing goods and rendering services Upload Your Academic
required for satisfying the needs and desire of our people in Nigeria. (Norbert Mille). Works and Start Earning
Government Through its various policies has given active encouragement and incentives Cash.
to small scale enterprises in form of banning the inflow of some foreign goods and
sometimes encouraging the importation of certain needed materials and equipment for the
private sector of the economy. In 1987, Government also went further to establish a
programme known as (NDE) National Directorate of Employment in Nigeria aiming at
precisly creating more job opportunities. It also served as a vehicle for promoting
sneurship amona the nation vouths. At inception NDE develoned four core 
programmes which is the youth employment and vocational skills, small scale enterprise
and Graduate employment, special public works, Agricultural and Rural employment
promotion.The youth employment and vocational skills development programme was further split
into four schemes namely:- National open Apprenticeship (NOA) scheme, schools on
wheels (SON) scheme, waste to wealth (WIO) scheme and Resettlement Scheme. Various Download
schemes that were embarked upon three of this programmes in one way or the others
encouraged self employment. They provide their participants with necessary vocational Download
and managerial skills, in order to cope with the demands of their preferred entrepreneurial
activities. They also provide them with part of their initial capital in form of guaranteed
and re-settlement loans as well as tools and equipments. Make Money
Government went on to set up other organizations for the development of small scale
enterprise from where entrepreneurs can obtain the necessary information and other Uefoete| Your Academic
helps they require in order to establish and run their business successfully. These Works and Start TEU]
institutions include: Co-operate Affairs Commission (CAC), Industrial Development Centre Cash.
(IDCS) Centre for Management Development (CMD), National economic Reconstruction
Fund (NEFRUND), Development banks and people bank of Nigeria (DPBN) etc. these
institutions went. Further to promote and develop small scale enterprises throughout the country.

\section*{Strategies for the Growth of Small-Scale Enterprise}
\title{ Expansion}
Expansion is one of the forms of internal growth of business. It means enlargement or increase in the same line of activity.
 Expansion is a natural growth of business enterprise taking place in course of time. In case of expansion, the enterprise grows
  its own without joining hands with any other enterprise. There are three common forms of business expansion.

ADVERTISEMENTS:

These are:

a. Expansion through Market Penetration:

It means the enterprise increases the sales of its existing product by enlarging the existing market. In other words, market
 penetration means making deeper in roads in the existing market. Various schemes are launched to penetrate into an existing market. The scheme for exchanging an old scooter for new one introduced by LML, for example, is a form of market penetration.

b. Expansion through Market Development:

ADVERTISEMENTS:

It implies exploring new markets for the existing product. In order to increase the sale of existing product, the enterprise
 makes searches for new customers.

c. Expansion through Product Development and/or Modification:

It implies developing or modifying the existing product to meet the requirements of the customers. Introduction of plastic bottles
 for selling refined oil in addition to lose sales is an example of product development /modification.

Advantages:

ADVERTISEMENTS:

Expansion provides the following advantages:

(i) Growth through expansion is natural and gradual.

(ii) Enterprise grows without making major changes in its organizational structure.

(iii) Expansion makes possible the effective utilization of existing resources of an enterprise.

ADVERTISEMENTS:

(iv) Gradual growth of enterprise becomes easily manageable by the enterprise.

(v) Expansion results in economies of large-scale operations.

Disadvantages:

However, against above advantages are disadvantages as well.

ADVERTISEMENTS:

These are:

(i) Growth being gradual is time consuming.

(ii) Expansion in the same line of product delimits enterprise growth making enterprise unable to take advantages from new business 
opportunities.

(iii) The use of modem technology is limited due to the limited resources at the disposal of enterprise. It weakens the competitive
 strength of the enterprise


\title{ Diversification}
Diversification is the most common form of internal growth of business. As mentioned above, expansion has its own limitations of 
business growth. Diversification is evolved to overcome the limitations of business growth through expansion. A business cannot 
grow beyond a certain point by concentrating on the existing product/market only.

In other words, it is not always possible for a business to grow beyond a certain point through market penetration. This underlines
 the need for the adding the new products / markets to the existing one. Such an approach to growth by adding new products to the 
 existing product line is called ‘diversification’.

In simple terms, diversification may be defined as a process of adding more products/markets/services to the existing one. This is 
necessary because, according to product ‘lifecycle concept’, every product has a definite life period. Like human beings, product 
also dies/disappears from the market. Hence, the introduction of new products to the basic product line becomes necessary to keep 
the business on.

ADVERTISEMENTS:

The use of diversification as a growth strategy has been continuously on increase both in the private and public sectors. 
In the private sectors, Kelvinator India Limited which was originally a refrigerator manufacturer diversified its product 
line into mopeds.

Similarly, Larsen and Toubro  an engineering company, diversified into cement. LIC’s diversification into mutual funds and 
SBI’s merchant banking are the examples of diversification adopted by the public sector in India.

Advantage:

Diversification offers the following advantages:

(i) Diversification helps an enterprise make more effective use of its resources.

(ii) Diversification also helps minimize risk involved in the business.

ADVERTISEMENTS:

(iii) Diversification adds to the competitive strength of the business.

(iv) Diversification also enables an enterprise to tide over business fluctuations and, thus, ensures smooth running of the business.

Disadvantages:

All is not good with diversification. It also suffers from certain disadvantages.

(i) Diversification involves business reorganization which requires additional resources. Thus, diversification becomes a 
costly proposition.

(ii) It becomes difficult, is not impossible, to effectively manage and coordinate the diverse business.

ADVERTISEMENTS:

Types of Diversification:

There is no uniform type of diversification adopted by all enterprises. It varies from enterprise to enterprise.

Usually, diversification is of four types:

a. Horizontal Diversification

b. Vertical Diversification

c. Concentric Diversification, and

ADVERTISEMENTS:

d. Conglomerate Diversification

A brief description of these follows:

a. Horizontal Diversification:

In this type of diversification, the same type of product or market added to the existing ones. Adding refrigerators to its
 original products of steel safes and locks by Godrej is an example of Horizontal Diversification.

b. Vertical Diversification:

In this type of diversification, complementary products or services are added to the existing product or service line of the 
enterprise. The new products or services serve either as inputs or a customer for the firm’s own product. A T.V. manufacturer may start producing picture tubes needed by it.

ADVERTISEMENTS:

Similarly, a sugar mill may develop a sugarcane farm to supply raw material or inputs for it. Setting up of retail shops by 
companies like Delhi Cloth Mills to sell its fabrics is also vertical type of diversification.

c. Concentric Diversification:

In case of concentric type of diversification, an enterprise enters into the business related to its present one in terms of 
technology, marketing or both. Nestle, originally, a baby food producers entered into related products like ‘Tomato Ketchup’ 
and ‘Maggi Noodles’. Similarly, a tea company like Lipton may diversify into coffee.

d. Conglomerate Diversification:

This type of diversification is just contrary to concentric diversification. In this type of growth strategy, an enterprise
 diversifies into the business that is not related to its existing business neither in terms of technology nor marketing. 
 JVG carrying on business in newspaper and detergent cake and powder, Godrej manufacturing steel safes and shaving cream 
 are examples of conglomerate diversification.

\title{ Joint venture}
Joint venture is a type of external growth strategy adopted by business firms. A joint venture could be considered as an entity 
resulting from a long-term contractual agreement between two or more parties, to undertake mutually beneficial economic activities,
 exercise joint control and contribute equity and share in the profits or losses of the company.

The Reserve Bank of India (RBI) has defined joint venture in the technical sense as: “a foreign concern formed, registered or 
incorporated in accordance with the laws and regulations of the host country in which the India party makes a direct investment, 
whether such investment amounts to a majority or minority shareholding.”

In simple terms, joint venture is a restricted or a temporary partnership between two or more firms to undertake jointly to 
complete a specific venture. The parties which enter into agreement are called co-ventures and this joint venture agreement
 will come to an end on the completion of the work for which it was formed.

The co-ventures participate in the equality and operations of the venture/ business. The profits or losses are shared between 
the co-ventures in their agreed ratio and in the absence of such agreement; the profits or losses are shared equally by the 
parties. In general, joint venture is formed for the purpose of consigning the goods from one place to another, undertaking
contracts for construction works, underwriting of shares or debentures of joint stock companies, etc.
Conditions for Joint Venture:

Joint venture may be useful to gain or access new business under some conditions, but not confined to the following only:

(i) When an activity is uneconomical for an organization to do alone.

(ii) When the risk of business has to be shared and, therefore, is reduced for the participating firm.

(iii) When the distinctive competency of two or more organizations can be brought together.

(iv) When setting up of an organization requires surmounting hurdles such as import quotas, tariffs, nationalistic-political 
interests and cultural roadblocks.

It is seen from above mentioned conditions that joint ventures are effective business growth strategy when the development costs 
have to be shared, business risk spread out and different expertise’s combined to make effective use of available resources and 
create synergy for outcomes.

Based on past experiences in the field of joint ventures, following five triggers have been identified to make joint ventures 
effective and successful:

i. Technology:

The foreign partner involved in joint venture can bring with it high-level technology, on the one hand, and the Indian counter
 partner provides good knowledge about the (local) market, on the other. The recent joint ventures taken place in the field of 
 telecom and automobiles are such examples.

ii. Geography:

When India has to compete in the larger and global market and a foreign player is already in a very commanding presence in the
 global market, this becomes a good trigger for joining a joint venture. One such example is insurance players such as Prudential
  and Standard Life having global presence. Thus, it becomes a good opportunity for the Indian partner to join such global partner
   in the joint venture.

iii. Regulation:

Regulation becomes a trigger especially when a sector which was highly restricted and closed sector for foreign partner for long
period is now opened up. Here again, insurance sector in India is one such example which was recently opened for foreign players.
It is due to this regulatory change the Indian partners like Bajaj and Indian Credit and Investment Corporation of India (ICICI)
joined foreign players in joint ventures in the insurance sector.

iv. Sharing of Risk and Capital:

This includes capital-intensive sectors like heavy-engineering requiring highly sophisticated technological expertise. In such cases
, both the partners involved in joint venture share risks and capital equally to effectively run the venture.

v. Intellectual Exchange:

Legal business could be such sector where both the partners gain intellectual advantage irrespective of law on the entry of foreign
 law firms in one country.

Types of Joint Ventures:

Experience suggests that joint venture is especially useful for entering international markets. As such, an Indian organization 
can enter a foreign market in a joint venture with a foreign organization. Similarly, a foreign firm can also enter into a joint
 venture with an Indian organization.

From the point of view of Indian organizations, the following five types of joint ventures are possible to form:

a. Between two Indian organizations in one industry:

Example is a joint venture between National Thermal Power Corporation Ltd. (NTPC) and the Indian railways for setting up a Rs.
 5,352 crore thermal power plant at Nabinagar in Bihar to meet the requirements of the rail network across the country.

b. Between two Indian organizations across different industries:

Example is a joint venture between Action Aid India (AAI) and Tata Institute of Social Sciences (TISS) to offer degree courses for
 rural communities in India.

c. Between an Indian organization and a foreign organization in India:

Example is joint venture with 50:50 between DLF Ltd. and Nakheel, a large property developer of the United Arab Emirates (UAE) for
 developing two integrated townships in India.

d. Between an Indian organization and a foreign organization in a third foreign country:

Example is a joint venture between Kirloskar Brothers Ltd. and SPP Pumps Ltd., United Kingdom (UK) for catering to the European Union
 (EU) market.

e. Between an Indian organization and a foreign organization in a third country:

Example is a joint venture between Apollo Tyres of India and Continental AG of Germany for setting up a tyre manufacturing joint 
venture in Malaysia.

Advantages:

The main Advantages the joint venture offers are as follows:

(i) Joint venture reduces risk involved in business.

(ii) It helps increase competitive strength of the business.

(iii) It makes possible the use of advanced technology and knowhow not available within a firm.

(iv) Joint venture provides the benefits of economy of scale by reducing production and marketing costs, on the one hand, and by 
increasing sales volumes, on the other.

Disadvantages:

Joint ventures suffer from the following disadvantages also:

(i) In case of lack of proper understanding between the co-ventures, the functioning of the business is adversely affected.

(ii) Excessive legal restrictions on foreign investments limit joining hands with foreign firms.

(iii) Sometimes, more equity participation by one or more co-ventures creates conflicts between them.

Reasons for Failure of Joint Ventures:

History of joint ventures reveals that there is a high probability of the joint ventures not working to the advantage of India. 
Therefore, this is suggestive that Indian organizations need to be on guard to save themselves from the disadvantages of joint
 venture arrangements.

Research studies report that the following reasons more often than not lead joint ventures to failure:

i. Change of Strategy:

India could cease to be interest of foreign organization for business alliance. For example, this has already happened with some
 foreign organizations like Bell Canada where Asia was considered as a market of no strategic significance.

ii. Regulatory Changes:

This is because of business laws in practice in the countries. For example, if the limit of Foreign Direct Investment (FDI) is 
kept at a low level and has not been raised. To quote, the FDI limit fixed at 26 percentage for some time now made foreign partners hesitant to form alliance in the Indian insurance sector.

iii. Success of Joint Venture:

Evidences are available to believe that if the joint venture is doing well, one of the alliance partners demands for increasing
 its share/holding in the joint venture. If not agreed by the other partner, joint venture arrangement comes to disband.

iv. Lack of Transparency:

In case one of the partners hides some facts or gives falsified facts, it causes confrontation and conflicts between the parties. 
If the conflict is not resolved, it may lead to break-up of business alliance. For example, the break-up of the Hutchison-Essar joint venture is one where the lack of transparency has been one of the key reasons. 

\title{Mergers and Acquisitions}
Merger and acquisition are yet other forms of external growth strategy. Merger means a combination of two or more existing
 enterprises into one. For the enterprise which acquires another, it is called ‘acquisition.’ For the enterprise which is acquired, it is called ‘merger.’ Thus, merger and acquisition are the two sides of the same coin.

If both organizations dissolve their identity to create a new organization, it is called consolidation. The other terms 
used for M and A are absorption, amalgamation, and integration. M and A are more popularly known as takeovers. For more than three decades after Independence, the normal route of growth was through licensing and setting up new projects.

But the post- liberalization, since 1991, has witnessed an increasing use of takeover strategies as the means or rapid growth.
 Mahindra  and  Mahindra’s takeover of a German company Schoneweiss, Tata’s takeover of Corus, and PricewaterhouseCoopers’s
  takeover of Mumbai-based taxation company RSM Ambit are illustrative examples of mergers  and  acquisitions.

Reasons for Mergers and Acquisitions:

Following are the illustrative ones:

Reasons for Buyer to Merge:

(i) To increase the value of the enterprise’s stock.

(ii) To increase the growth rate and make a good investment.

(iii) To improve the stability of its earnings and sales.

(iv) To balance, compete or diversify its product line.

(v) To reduce competition.

(vi) To acquire a needed resource quickly.

(vii) To avail tax concessions and benefits.

(viii) To take advantage of synergy.

Reasons for Seller to Merge:

(i) To increase the value of the owner’s stock and investment.

(ii) To increase the growth rate.

(iii) To acquire resources to stabilize resources.

(iv) To benefit from the tax legislation.

(v) To deal with top management succession problem.

Types of Mergers and Acquisitions:

Mergers and acquisitions can be classified into the following types:

a. Horizontal M and A:

Horizontal M and A take place when there is a combination of two or more organizations in the same business, 
or organizations engaged in certain aspects of the production or marketing processes. A footwear company combining 
with another footwear company is one such example of horizontal M and A.

b. Vertical M and A:

In vertical M and A, two or more organizations, not necessarily in the same business, come together to create 
complementarities either in terms of supply of materials (say material) or marketing of goods and services (say outputs).
 For example, pharmaceutical company combines with retail medical store.

c. Concentric M and A:

This refers to two or more organizations related to each other either in terms of customer functions or alternative technologies
 combine together. For example, a footwear company combines with a hosiery firm making socks.

d. Conglomerate M and A:

This is just opposite of concentric M and A. In this case, two or more organizations not related to each other either in 
terms of customer functions or alternative technologies. Combination between a pharmaceutical company and footwear company
 is one such example.

Advantage:

Mergers and acquisitions provide the following advantages:

(i) Provide benefits of economies of scale in terms of production and sales.

(ii) Facilitate better use of resources.

(iii) Enable sick enterprises to merger into the healthy ones.

(iv) Promote diversification in product line to take advantages of opportunities available in the particular business.

Disadvantages:

Mergers and acquisitions are not unmixed blessings.

These also suffer from the following drawbacks:

(i) Larger scale operations often make co-ordination and control ineffective. This adversely affects business performance as a whole.

(ii) Sometimes mergers and acquisitions lead to monopoly in the particular business. Monopoly is not welcome in the interest 
of the society.

Important Issues Involved in Mergers and Acquisitions

Mergers and acquisitions are as much important are not so simple. Meaningful mergers and acquisitions involve expertise in
 special areas such as accounting, finance and legal matters and negotiations.

Following are some of the important strategic, financial, managerial, and legal issues involved in mergers and acquisitions:

a. Strategic Issues:

These issues relate to the commonality of strategic interests between the buyer and seller firms. The main objective of M and A 
is to create synergetic effects for the enterprises. Therefore, the strategic advantages and distinctive competencies due to 
M and A for the merging enterprises have to be duly examined and analysed.

It is also important to note that there has to be a fine match between the objectives of the firms involved in M and A. 
For example, a merger should ideally lead to the generation of sufficient strengths that would help the enterprise during 
the post-merger duration to achieve its objectives in an effective and better manner.

b. Financial Issues:

There are three major financial issues involved in M and A.

These are:

(i) Valuation of the business and shares of the target firm;

(ii) Sources of financing for mergers; and

(iii) Taxation matters after M and A.

The valuation of the business of the target firm is a detailed and comprehensive process that should take into account a 
range of factors including the tangible and intangible assets, the industry profile of the firm and its prospects and the 
future earnings and prospects of the target firm.

Similarly, the valuation of the shares in an M and A is equally complicated process involving issues such as the stock exchange
 price of the shares of the target firm, dividends paid, growth prospects of the firm, value of its assets, quality and integrity
  of the top management, competitive conditions, opportunity costs in terms of investments and market sentiments.

The second financial issue is of the sources of financing required for enterprises involved in M and A. Several sources of 
funds available range from the acquiring companies’ own funds or borrowed funds, raised through the issue of debentures, bonds,
 deposits, external commercial borrowings, global depositary receipts, loans from Central or State financial institutions or 
 rehabilitation finance provide to sick industrial companies.

The third issue is of the taxation matters that are dealt with under the relevant provisions of the Income Tax Act, 1961, and
 which are related to various technical aspects such as the carrying forward or set-off of losses and unabsorbed depreciation, 
 capital gains, tax and amortization of expenses.

c.. Managerial Issues:

These issues relate to the umpteen problems of managing enterprises after the M and A has taken place. It is important to note
 that the perception of how the management will take place after M and A also matters and affects the process involved in it. 
 The usual experience is that the post M and A is characterized by changes in staff, specially chief executives and top managers.

If there is an assurance that the merger will lead to a status quo, or that‘ professional management’ would be adopted,
 then the M and A process may take place smoothly. On the contrary, if the M and A is perceived as threatening, it results 
 in resistance and opposition by the various groups.

This happens because the post-merger period poses uncertainty to the managers of the merging organizations. The reason is that
 they feel insecure about their job, status within the organization, and their earnings and promotional prospects.

The consequence of feeling threatened by the impending changes due to M and A, the existing managers oppose change which, in
 turn, leads to low morale and productivity and often resulting in mass exodus of managers from the organization.

d. Legal Issues:

These issues relate to the provisions made in law for the purpose of M and A. In India, the provisions relating to M and A and other
 schemes are contained in Chapter V of the Companies Act, 1956 and specifically, in Sections 391 to 395 of the Companies Act, 1956 and
  in the rules 67 to 87 of the Companies (Court) Rules, 1959.

The implementation of the strategies of M and A requires a thorough understanding of relevant provisions. It is interesting to mention
 that the term ‘merger’ is not used in the Companies Act; only the term ‘amalgamation’ is used in Section 394 of the Act. 
 The only section that deals with the transfer of shares (or takeover bids) is Section 385.

Apart from the Companies Act and the MRTP Act, Section 72 A (I) of the Income Tax Act, 1961 is also relevant for taxation
 purposes of amalgamated companies and provides for carrying forward accumulated losses and unabsorbed depreciation of the
  amalgamating company, i.e. M and A organizations.

How Mergers and Acquisitions take place?

M and A can take place in various ways. There is no specific and standard procedure available for M and A to take place. 
However, based on experiences relating to M and A, it is realized that following certain guidelines can be useful for M and 
 As to take place systematically.

The major steps include but are not limited to the following only:

a. Spell out the objective

b. Indicate how the objective would be achieved

c. Assess managerial quality

d. Check the compatibility of business styles

e. Anticipate and solve problems early

f. Treat people with dignity and concern 

\end{document}